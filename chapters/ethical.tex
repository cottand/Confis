% Chapter Template

\chapter{Ethical Considerations}\label{ch:ethical}


\section{Legal Implications}\label{sec:legal-implications}

\subsection{Possible Discrepancies Between Machine-Readable Representations and Legal Requirements}
\label{subsec:legal-discrepancies}

This project hopes to automate enforcement of clauses in legal agreements.
Contracts commonly have exceptions where violating the terms of the agreement is not a breach of the
contract in circumstances outside the control of all parties (referred to as \textit{`Force
Majeure'}~\cite{forceMajeureDefinition} - see~\cite[\textsection13.14]{jetbrainsEduLicence} for an
example).\\

% TODO come back to this after writing the intro
It is still to be seen to what extent this project enforces clauses and to what extent it only
provides a decentralised audit trail.
The former may impede actions that could be law-compliant in case of force majeure, while the latter
would give parties more freedom to breach the contract (and leave it up to the other party to follow
up with legal action).

While the latter may seem like a safer option, enforcing compliance outside of courts can be very
attractive to all parties because they would be able to protect the terms of the contract without
having to go to court and expending the financial and legal resources to take action in case of
contract breach.
So much so that this is the case already in many software licenses~(where~\cite{jetbrainsEduLicence}
is a prime example, see~\ref{subsec:licensing:software}), where the end user has their access
revoked before the licensor.

% TODO maybe there is better wording for this term

\subsection{Legal Validity}

If this project wants to represent or encode legal agreements in any way, they should qualify as
such within the legal definitions of a legal system (or of several!).\\

Ideally, a court should be able to recognise a machine-readable representation as well as a
traditional manually typed one.
This must also hold true for other elements such as signatures (digital signatures as opposed
handwritten ones), written notices, etc.


\section{Misuse}\label{sec:misuse}

Depending on the future scope of this project, one of the parties could abuse self-executing
contract clauses (for example, to revoke access to data in breach of the terms of an agreement).

Different possible thread models will have to be carefully considered taking into account scenarios
with malicious vs honest-but-curious adversaries.


\section{Data Privacy and Compliance}\label{sec:data-privacy-compliance}

Legal agreements between businesses can be confidential both in their contents and their existence -
the project should aim to preserve these properties, whatever the format it considers for
representing such agreements.\\

As for General Data Protection Regulation (GDPR)~\cite{gdprInfo}, if this project's prototype is
implemented as software libraries (rather than a centralised server under control of a third party),
then compliance obligations should remain within the responsibilities of the party using the
software~\citeTODO.


\section{Dual Use}\label{sec:dual-use}

This project has an exclusively civilian focus.


\section{Environmental Implications}\label{sec:environmental-implications}

%TODO more citations here
While this project does not consider directly using any environmental-unfriendly technology, it
considers using decentralised ledgers which heavily rely on Proof-of-Work
algorithms~(see~\ref{subsec:btc:pow}) which has shown to be energy-hungry when used at the scale of
Bitcoin or Ethereum~\cite{GOODKIND2020101281}.\\

While addressing these issues is outside the scope of this project, at the time of writing the
Ethereum community aims to make significant changes to its consensus mechanism (by moving away from
proof-of-work altogether) that should greatly mitigate this environmental impact~\cite{eth2Vision}.
