\chapter{Background}\label{ch:background}


\section{Cryptography Primitives}\label{sec:cryptography}


\section{Bitcoin Protocol}\label{sec:bitcoin}

Blockchain technology was introduced by~\cite{nakamoto2008bitcoin} as a decentralised system allowing for electronic
cash payments.
Blockchains are immutable distributed ledgers where participants' balances can be verified by every other participant,
and it is computationally hard to tamper with balances to perform attacks (such as performing a transaction where a
participant spends more funds than what they own). \\
I will provide a brief overview of how Bitcoin provides these guarantees.

\subsection{Transactions}\label{subsec:btc:txs}

\cite{nakamoto2008bitcoin} defines an \textit{electronic coin} as a chain of signatures: a payer can use their private key, the
hash of the previous transaction, and the payee's public key to create a signed hash that can be verified by the payee
(and used by them for \textit{their} next transaction).
This is illustrated in Figure~\ref{fig:bitcoin-tx}.




\begin{figure}[th]
    \centering
    \includegraphics[width=0.8\columnwidth]{figures/bitcoin-tx}
    \decoRule
    \caption[Bitcoin coin ownership transfer]{Transfer of ownership signature chain, from~\cite{nakamoto2008bitcoin}}
    \label{fig:bitcoin-tx}
\end{figure}

This ensures that, as long as a participants sign transactions at most once:
\begin{itemize}
    \item By verifying the chain of signatures, every participant can verify which participant owns which coin
    \item Only the owner of a coin can initiate a transaction with that coin
\end{itemize}

Bitcoin enforces that participants can only sign transactions once thanks to its proof-of-work (see~\ref{subsec:btc:pow}) algorithm.

\subsection{Proof-Of-Work}\label{subsec:btc:pow}

Bitcoin ensures 'unique signatures' in transactions by grouping transactions in immutable, public \textit{blocks}.
Participants can then verify a payer has not signed a hash of a single transaction twice by looking at all existing
transactions.\\
Blocks are made immutable by including in them a value (called a \textit{nonce}) and the hash of the previous block.
% TODO verify it is the hash of the entire block that must yield the zeroes (implementation detail really)
The protocol then accepts only blocks where the $n$ first bits of its hash are zeroes. \\

Thus, in order to publish a block a participant must do work to find a nonce such that the block's hash meets this
condition - then other participants can verify its validity with a single hash operation.
This guarantees that a block cannot be changed (ie, a new copy published) without redoing the computational work.
Because blocks are chained (they include the hash of the previous block), in order to modify a transaction in the past
an adversary needs to redo the computational work for every block since that transaction.
% TODO implementation of mining rewards
Additionally, participants that successfully find a suitable nonce and propose new blocks (also referred to as
\textit{miners}) are allowed to add a specific transaction to the block where they own a newly created coin (also
referred to as \textit{mining reward}).

This model of consensus ensures that
\begin{itemize}
    \item Participants have a monetary incentive to stay honest with respect to the protocol
    \item An honest chain will out-compete an adversary's chain as long as the majority of computing power is honest
\end{itemize}

\subsection{Further details of the Bitcoin protocol}\label{subsec:btc.details}

While I provide a high-level overview of what makes the protocol function, there are many more details that combined
allow for more efficiency and usability:
\begin{itemize}
    \item Transactions may have several inputs and outputs, so participants can transfer amounts rather than single
    \textit{electronic coins}.
    Thus, a participant's balance is the sum of all the unspent outputs of previous transactions.
    \item By modifying how many of the leading bits of a blocks' hash must be zeroes, the average computation necessary to
    produce a new block can be adjusted by the protocol.
    \item By using Merkle Trees~\cite{merkle1980tree} transactions with fully spent transaction outputs can be discarded
    without breaking the block's hash.
    This allows compacting old blocks to reclaim disk space.
    \item A participant that does not wish to mine or hold a copy of the entire blockchain can still verify payments.
    It can keep a copy of the block headers of the longest chain and link a transaction to where it is on-chain and
    check that other network nodes have accepted it.
\end{itemize}
\\
For more information on all the workings of the Bitcoin protocol, please refer to~\cite{nakamoto2008bitcoin}.


\section{Ethereum Smart Contracts}\label{sec:ethereum}
% TODO idk to what extent I will need this?


\section{Licensing}
% TODO

\subsection{Dataset Licesing}
% TODO


\section{Zero-Knowledge Proofs}
% TODO

