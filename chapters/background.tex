\chapter{Background}\label{ch:background}

\section{Bitcoin Protocol}\label{sec:bitcoin-protocol}

Blockchain technology was introduced by~\cite{nakamoto2008bitcoin} as a decentralised system allowing for electronic
cash payments.
Blockchains are immutable distributed ledgers where participants' balances can be verified by every other participant,
and it is computationally hard to tamper with balances to perform attacks (such as performing a transaction where a
participant spends more funds than what they own). \\
I will provide a brief overview of how Bitcoin provides these guarantees.

\subsection{Transactions}

\cite{nakamoto2008bitcoin} defines an \textit{electronic coin} as a chain of signatures: a payer can use their private key, the
hash of the previous transaction, and the payee's public key to create a signed hash that can be verified by the payee
(and used by them for \textit{their} next transaction).
This is illustrated in Figure~\ref{fig:bitcoin-tx}.




\begin{figure}[th]
\centering
\includegraphics[width=0.8\columnwidth]{figures/bitcoin-tx}
\decoRule
\caption[Transaction signing in Bitcoin]{Transaction signing in Bitcoin, from~\cite{nakamoto2008bitcoin}}
\label{fig:bitcoin-tx}
\end{figure}

This ensures that, as long as a participants sign transactions at most once:
\begin{itemize}
    \item By verifying the chain of signatures, every participant can verify which participant owns which coin
    \item Only the owner of a coin can initiate a transaction with that coin
\end{itemize}

Bitcoin enforces that participants can only sign transactions once thanks to its proof-of-work (see~\ref{subsec:btc:pow}) algorithm.

\subsection{Proof-Of-Work}\label{subsec:btc:pow}

Bitcoin ensures 'unique signatures' in transactions by grouping transactions in immutable, public \textit{blocks}.
Participants can then verify a payer has not signed a hash of a single transaction twice by looking at all existing
transactions.\\
Blocks are made immutable by including in them a value (called a \textit{nonce}) and the hash of the previous block.
% TODO verify it is the hash of the entire block that must yield the zeroes (implementation detail really)
The protocol then accepts only blocks where the hash's $n$ first bits are zeroes. \\

Thus, in order to publish a block a participant must do work to find a nonce such that the block's hash meets this
condition, and other participants can verify its validity with a single hash operation.
This guarantees that a block cannot be changed (ie, a new copy published) without redoing the computational work.
Because blocks are chained (they include the hash of the previous block), in order to modify a transaction in the past
an adversary needs to redo the computational work for every block since that transaction.



\section{Ethereum Smart Contracts}


Lorem ipsum dolor sit amet, consectetur adipiscing elit. Aliquam ultricies lacinia euismod. Nam tempus risus in dolor rhoncus in interdum enim tincidunt. Donec vel nunc neque. In condimentum ullamcorper quam non consequat. Fusce sagittis tempor feugiat. Fusce magna erat, molestie eu convallis ut, tempus sed arcu. Quisque molestie, ante a tincidunt ullamcorper, sapien enim dignissim lacus, in semper nibh erat lobortis purus. Integer dapibus ligula ac risus convallis pellentesque.

\subsection{Subsection 1}

Nunc posuere quam at lectus tristique eu ultrices augue venenatis. Vestibulum ante ipsum primis in faucibus orci luctus et ultrices posuere cubilia Curae; Aliquam erat volutpat. Vivamus sodales tortor eget quam adipiscing in vulputate ante ullamcorper. Sed eros ante, lacinia et sollicitudin et, aliquam sit amet augue. In hac habitasse platea dictumst.


\subsection{Subsection 2}
Morbi rutrum odio eget arcu adipiscing sodales. Aenean et purus a est pulvinar pellentesque. Cras in elit neque, quis varius elit. Phasellus fringilla, nibh eu tempus venenatis, dolor elit posuere quam, quis adipiscing urna leo nec orci. Sed nec nulla auctor odio aliquet consequat. Ut nec nulla in ante ullamcorper aliquam at sed dolor. Phasellus fermentum magna in augue gravida cursus. Cras sed pretium lorem. Pellentesque eget ornare odio. Proin accumsan, massa viverra cursus pharetra, ipsum nisi lobortis velit, a malesuada dolor lorem eu neque.

\section{Main Section 2}

Sed ullamcorper quam eu nisl interdum at interdum enim egestas. Aliquam placerat justo sed lectus lobortis ut porta nisl porttitor. Vestibulum mi dolor, lacinia molestie gravida at, tempus vitae ligula. Donec eget quam sapien, in viverra eros. Donec pellentesque justo a massa fringilla non vestibulum metus vestibulum. Vestibulum in orci quis felis tempor lacinia. Vivamus ornare ultrices facilisis. Ut hendrerit volutpat vulputate. Morbi condimentum venenatis augue, id porta ipsum vulputate in. Curabitur luctus tempus justo. Vestibulum risus lectus, adipiscing nec condimentum quis, condimentum nec nisl. Aliquam dictum sagittis velit sed iaculis. Morbi tristique augue sit amet nulla pulvinar id facilisis ligula mollis. Nam elit libero, tincidunt ut aliquam at, molestie in quam. Aenean rhoncus vehicula hendrerit.
Stand on
