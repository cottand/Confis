\chapter{A Domain Specific Language for Legal Agreements}\label{ch:lang}

One of the key focuses of this project is not just to be able to represent legal contracts with specific properties and capabilities as described in~\autoref{ch:queries};
but also to make it as easy as possible for people with non-technical backgrounds to use such representations.
Simply put, a lawyer should not need to learn JSON.\\

This is the main motivation for developing tooling which aims to make it easier for people of non-technical backgrounds to produce, modify, and understand Confis legal agreement representations.
The core of this tooling is the Confis language~(\autoref{sec:developing-a-dsl}) and an IDE-assisted editor~(\autoref{sec:additional-dsl-tooling}).


\section{Developing a DSL}\label{sec:developing-a-dsl}

\subsection{Requirements and Design Decisions}\label{subsec:dsl:requirements}

\subsubsection{Easy to write while still machine-readable}

Writing a Confis agreement should be close to writing plain English, while still being machine-readable.
For more details on the meaning of \emph{machine-readable} in the context of this project, see~\autoref{sec:machine-readable-contracts}.\\


A compromise must be struck between ease of writing and machine-readability.

\paragraph{Data Serialization Language}

On one extreme, a data serialization text file (like JSON, YAML, or XML) would allow writing text that can be easily parsed by a program.
But writing such files requires some data structures knowledge;
and because of their key-value nature they do not allow writing sentences, leaving them too far from the readability of human-written legal prose.

\paragraph{Natural Language Processing}

On the other extreme, legal prose processed through a language processing program allows the drafter to completely ignore the machine-readable aspect of the document.
Readers would be able to integrate such documents in their existing workflows -- as they would need to make no transition from their existing, non-machine-readable documents.
This is the approach discussed in~\autoref{sec:nlp}.\\

A compromise between these two solutions would be a language formal enough that it can be parsed by a program, but natural enough that natural language sentences can be recognised in it.
Python or AppleScript~\cite{Sanderson2010appleScript}~(see~\autoref{fig:appleScript}) are good examples of programming languages (and therefore parseable) that are engineered with the goal of resembling English as much as possible.

\begin{figure}[h]
    \centering
    \begin{verbatim}
                set the firstnumber to 1
                set the secondnumber to 2
                if the firstnumber is equal to the secondnumber then
                    set the sum to 5
                end if
    \end{verbatim}
    \caption{Sample code snippet of the AppleScript Language~\cite{Sanderson2010appleScript}}
    \label{fig:appleScript}
\end{figure}

This project therefore proposes developing a DSL~(see~\nameref{sec:dsls}) as a suitable compromise that allows working with a human-readable encoding which can then be compiled to a suitable machine-readable representation.\\

We will call this language \emph{Confis DSL}, or \emph{Confis language}, and the machine-readable representation it compiles to \emph{Confis Internal Representation} (or \emph{Confis IR}).
The Confis IR will be used for processing as discussed in~\autoref{ch:queries}.

\subsubsection{Easy to develop and extend}

While compiler development and the challenges it brings are a very interesting field of \subjectname, this project does not aim to make contributions in this area.
It hopes to be more concerned with introducing a suitable abstraction that allows both drafting and processing legal documents.


\section{Additional Editing Tooling}\label{sec:additional-dsl-tooling}

aa.
