% Chapter Template

\chapter{Evaluation}\label{ch:evaluation}


\section{Confis Project Goals}\label{sec:eval:goals}

This project hopes to leverage software engineering techniques (like domain-specific languages), existing logic abstractions (like normative rules), and existing technologies relating to law (like Ricardian contracts) in order to provide a framework allowing people with non-technical and non-legal backgrounds to draft, review, and access information in legal agreements.

The following criteria are motivated by how existing technologies in the state-of-the-art (see sections~\ref{sec:nlp} and~\ref{sec:machine-readable-contracts}) fail to address them:

\begin{definition}[Meaningful Representation]
    \label{def:meaningful-representation}
    Confis should be able to fully represent a legal agreement in a legal context.

    That is, (albeit with the use of tooling) a Confis agreement should be usable in place of a legal one.

\end{definition}

\begin{definition}[Accessibility]
    \label{def:accessibility}
    Whatever the encoding of a Confis agreement, should be possible to produce (albeit with the use of tooling) such Confis agreement without an in-depth understanding of the formalism that encodes the agreement.
\end{definition}

The following are loosely defined -- this is unavoidable because this project aims to introduce a formalisation for legal documents, which are not well-defined.
Please see~\nameref{sec:language-semantics} for the definitions of~\nameref{def:action},~\nameref{def:party},~\nameref{def:permission}, and~\nameref{def:requirement}.

% TODO revisit after writing reasoning section
\begin{definition}[Completeness]
    \label{def:completeness}
    A Confis agreement $C$ is complete with respect to a legal agreement $L$ if it represents $L$ and:

    \begin{itemize}
        \item If $L$ allows a legal capability $c$, then $C$ also allows $c$ through an equivalent permission
        \item If $L$ forbids a legal capability $c$, then $C$ also forbids $c$ through an equivalent permission
        \item If $L$ has a requirement $r$, then $C$ has $r$
    \end{itemize}
\end{definition}


\begin{definition}[Soundness]
    \label{def:soundness}
    A Confis agreement $C$ is sound with respect to a legal agreement $L$ if it represents $L$ and:

    \begin{itemize}
        \item If $C$ has an \texttt{Allow} permission $p$ equivalent to a capability $c$, then a $L$ also allows $c$
        \item If $C$ has a \texttt{Forbid} permission $p$ equivalent to a capability $c$, then a $L$ also forbids $c$
        \item If $C$ has a requirement $r$, then $L$ has $r$
    \end{itemize}
\end{definition}


\section{Language Formalism Evaluation}\label{sec:language-formalism-evaluation}

The Confis language formalism and semantics are discussed in~\autoref{sec:language-semantics}.
This project will evaluate it in how it meets the requirements described in~\autoref{sec:eval:goals}, its expressiveness (how capable it is to represent complex contracts) as well as how it performs in relation to comparable formalisms (like~\nameref{subsubsec:symboleo}~\cite{symboleo2020} or~\nameref{subsec:accord}~\cite{accordHomepage}).

As far as fulfilling the~\nameref{def:accessibility} requirement, Confis is the only formalism in the literature that makes an effort to penetrate industry by
\begin{itemize}
    \item Trying to avoid requiring a scientific or engineering background from its users.
    \item Providing additional tools to ease development and shorten the iteration loop.
\end{itemize}

The Query UI~(\autoref{sec:queryUI}), Confis-to-English conversion~(\autoref{sec:additional-tooling:doc-rendering}), the Confis Editor~(\autoref{sec:confis-editor}) and the structure of the Confis language itself (\autoref{sec:language-semantics}) were all developed with this goal in mind.

Exceptions to this statement include accessible technologies like Juro (\autoref{subsec:juro}).
Such tools add contract metadata without actually allowing for querying beyond fetching the metadata, nor attempt to capture the semantics of the agreement -- like Adobe Signing tools or typical Ricardian Contracts~\cite{ricardianWeb}, but unlike Symboleo or The Accord Project.\\

We translate to Confis a sample contract used in Symbolio~\cite{symboleo2020} in order to compare the same agreement in two different languages.
For the sake of brevity, we wille examine an extract, but the full original (in plain English) can be found at~\autoref{tab:meat}, the full Symbolio specification at~\autoref{fig:symbolio:meatSales}, and the full Confis agreement at~\autoref{fig:confis:meat}.

\begin{table}[h]
    \centering
    \setlength{\fboxsep}{10pt}
    \fbox{
        \begin{minipage}{0.8\textwidth}
            \textbf{Confidentiality}
            \begin{enumerate}
                \item Both Seller and Buyer must keep the contents of this contract confidential during the execution of the contract and six months after the termination of the contract.
            \end{enumerate}
        \end{minipage}
    }
    \caption[Sample Confidentiality clause]{Sample confidentiality clause, extracted from~\autoref{tab:meat}}
    \label{tab:meat-confidentiality}
\end{table}

The comparison will be performed on a rather simple clause of the agreement, a confidentiality clause specified by~\autoref{tab:meat-confidentiality}.
The Confis representation is specified by~\autoref{fig:confis:meat-confidentiality}, while the Symbolio one is specified by~\autoref{fig:symbolio:meatSales-confidentiality}.
Both are re-written self-contained examples (rather than text extracts from the original, longer contracts).
This was done in order to fully reflect the necessary boilerplate and ceremony of each language.\\


\begin{figure}[h]
    \centering
    \begin{minted}[
        autogobble,
        frame=lines,
        framesep=2mm,
        fontsize=\small
    ]{kotlin}
val effDate = 1 of June year 2022
val reveal by action(
    description = "as in not keeping the contents confidential"
)
val contract by thing("the Contract", description = "this Agreement")
val seller by party("the Seller", description = "Alice Liddell")
val buyer by party("the Buyer", description = "The Meat Supermarket, Inc")

seller mayNot reveal(contract) asLongAs {
    within { effDate..(effDate + 6.months) }
}

buyer mayNot reveal(contract) asLongAs {
    within { effDate..(effDate + 6.months) }
}
    \end{minted}
    \caption{Confis for~\nameref{tab:meat-confidentiality}, extracted from~\autoref{fig:confis:meat}}
    \label{fig:confis:meat-confidentiality}
\end{figure}

\begin{figure}[h]
    \begin{minted}[
        autogobble,
        frame=lines,
        framesep=2mm,
        fontsize=\small
    ]{prolog}
        Domain meatSaleDomain
        Seller isA Role with returnAddress: String, name: String;
        Buyer isA Role with warehouse: String;
        Disclosed isAn Event;

        endDomain

        Contract MeatSale (buyer: Buyer, seller: Seller, effDate: Date)

        Declarations
        disclosed: Disclosed;

        Surviving Obligations
        so1 : Obligation(seller, buyer, true,
            not WhappensBefore(disclosed, Date.add(Activated(self), 6, months))
        );

        so2 : Obligation(buyer, seller, true,
            not WhappensBefore(disclosed, Date.add(Activated(self), 6, months))
        );
        endContract

    \end{minted}
    \caption{Symboleo Specification for~\nameref{tab:meat-confidentiality}, extracted from~\autoref{fig:symbolio:meatSales}}
    \label{fig:symbolio:meatSales-confidentiality}
\end{figure}

\paragraph{Specifying a legal Obligation}
Notice how Symboleo allows representing a more complex domain by specifying an \emph{Event} \texttt{`Disclosed'}, and constraining the legal capabilities of Buyer by creating an \emph{Obligation} (with a notion of this obligation being \emph{from} Buyer \emph{towards} Seller) which specifies that the \texttt{disclosed} event cannot happen before six months after the end of the contract.
In Symbolio's model if \texttt{disclosed} happen, the breach cannot be attributed to either Seller nor Buyer.

Confis specifies a simpler domain -- while it also specifies Seller and Buyer and represents `disclosing' as an Action, it has no notion of \emph{creditor} and \emph{lender} when it comes to Requirements (defined in Definition~\autoref{def:requirement}).
On the other hand, Confis can `blame' specific parties for a breach, as it attributes Actions to Subjects.
Confis is also unable to keep track of its own execution time -- instead it requires specifying the date in the contract.
Notice how this abstracts away the document from the real-world execution date, and how this limitation of Symboleo is omitted in the original paper~\cite{symboleo2020}, but present in the sample source code~\cite{symboleoMeat}.

Both models are capable of expressing periods of time, but in Confis they are not part of the algebra (they are instead more general~\nameref{def:circumstance}s).

\paragraph{Accessibility} While both contracts require language knowledge to write them, Confis can be read without prior training and its functions (\texttt{within}, \texttt{mayNot}, \texttt{asLongAs},~\dots) are easy to memorise.
The best example of this is the operation of summing six months to a time period -- while Symboleo requires specifying a \texttt{Date} library and wrapping the duration with a function (\texttt{`Date.add(..., 6, months)'}), Confis uses operator overloading to sum to a date (\texttt{`... + 6.months'}).

As far as readability goes, Confis goes the extra mile by trying to combine metadata and language semantics to provide an English-like preview.
This prose rendering is shown in~\autoref{fig:confis-render-meat-confidentiality}.
While the preview is not as clear as the original plain-English clause, it conveys the obligations of each party well and is unambiguous thanks to its definition referencing (which is common in real legal agreements).

\begin{figure}[h]
    \centering
    \includegraphics[width=0.7\columnwidth]{figures/confis.meat.prose}
    \caption{Confis prose rendering of~\autoref{tab:meat-confidentiality}}
    \label{fig:confis-render-meat-confidentiality}
\end{figure}


\section{Software Deliverables}\label{sec:software-deliverables}

\subsection{Language Implementation}

\subsection{Querying Engine}

\subsection{Tooling Implementation}

\section[Overall Evaluation]{Overall Evaluation With Respect to Goals}


\section{Machine Readable Licence Representations}\label{sec:licence-representations}

% we are looking to eavaluate the project - a lot of this should go into the intro probably...

% measures will be cost (smart contracts!), lawyer avoidance, and advice from people that know about this stuff

Other goals of this project require that a legal agreement can be encoded for it to become
structured data.
The project should come up with, or find in the literature and adapt, a suitable format for a legal
agreement.
In order to limit the scope of the project, we will narrow the agreement to be encoded to
licences.\\

A successful project should include:
\begin{itemize}
    \item \textbf{Machine-readable representation of licenses}.
    \item \textbf{Prototype Software} that uses the contract representation to ease the existing
    manual workflow related to agreements in a business.
    This could include a searchable database that is aware of licenses related to the same data or
    software~(see~\ref{sec:contract-registry}).
    The software should, to some extent, allow querying a contract to be able to determine the
    conditions which the license allows to use the data in.
    \item Secure, \textbf{self-executing clauses} that all parties bound to the contract can trust.
    This includes making it so parties cannot have plausible deniability or repudiation if they are
    dishonest.
\end{itemize}

Legal contracts are complex documents that may list vague or hard-to-encode conditions and
situations.
I expect not being able to (inexpensively) capture the entire meaning of a contract in a
machine-readable format.
This disassociation between reality and representation should be fully taken into account when
providing guarantees and making assumptions.
For example, a successful prototype should be aware of this enough to refer the user to the original
license when it is unable to provide certainties with respect to a query against a license.

In other words, this uncertainty created by attempting to represent a contract in a machine-readable
encoding should be part of the encoding itself, for the sake of completeness and usability.\\

Evaluation criteria would include whether the project can successfully encode a wide range of
licence agreements.


\section{A Universal Contract Registry}\label{sec:contract-registry}

A decentralised contract registry, shared by multiple businesses (not unlike a decentralised version
of Juro, see~\ref{subsec:juro}), can be a very useful contribution if it manages to meet the
requirement of privacy.
Businesses need not just the terms of their agreements to be secret to competitors
(see~\cite[\textsection2.1]{economistIU2016licence}), but also the existence of the contracts
themselves must remain private.
